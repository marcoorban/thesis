\documentclass[../main.tex]{subfiles}
\graphicspath{{\subfix{../images/}}}

\begin{document}
	
\chapter{Literature Review}

What I need to do in the literature review is an introduction, body and conclusion. The introduction should be an overview of the topics that I am going to talk about, and probably introduce the research motivation.

\section{Introduction}

Additive manufacturing, also known as 3D printing, is a manufaacturing process in which parts are built by stacking layers of material on top of each other until the desired geometry is created. In particular, powdered-bed fusion is a type of additive manufacturing in which the base material is composed of a metallic powdered. A bed is then filled with this powder, and a high-powered laser or electron beam is utilized to selectively add heat to locations of this powdered metal, thus melting it. As the melted portions cool, it leaves a layer of solid material, and the whole component is then built layer by layer in this manner. If a laser is used for manufacturing, the process is then known as laser powder bed fusion (LPBF).

Additive manufacturing in general allows for the creation of parts with complex geometries that would otherwise be difficult, time-consuming ortoo costly to manufacture with other more traditional means of manufacturing such as machining or casting. This can be contrasted with the manufacture of prothesis using traditional manufacturing, wchih relies on the high-volume production of standarized parts and shapes. Due to the ease of creating parts with complex geometries and with shorter lead times, additive manufacturing has had a growing interest from the medical industry for the creation of custom made implants tailored to an individual patient's anatomy \cite{marshTrendsDevelopmentsHip2021} \\cite{narraAdditiveManufacturingTotal2019}. Additionally, this benefit of creating customized parts also leads to an improvement of patient comfort and better outcomes in orthopedic and surgical applications, as well as enhacing surgical precision and reducing complications of post-operative care \cite{mobarakRecentAdvancesAdditive2023} \cite{pathak3DPrintingBiomedicine2023}.

Nevertheless, even though additive manufacturing holds promise for the production of cuztomized medical implants, its cost can still remain high due to machine, material and process-level expenses, and thus its usage can still be limited only for revision procedures, i.e. second or third procedures in which the objective is to replace a failed or uncomfortable component \cite{narraAdditiveManufacturingTotal2019}. To solve the problem of high material costs, Laureijs et. al. propose that, in the case of powdered base additive manfucturing, one of the main driving costs is the price of the powder price. One of the viable options to drive down the price of additive manufacturing then would be to reduce the amount of scrap material involved in the process. Scrap material in additive manufacturing results from discarding the support structures that the part requires as well as any addition or removal of material to the part in post-processing steps to ensure that the part is maintained withing tolerances \cite{MetalAdditiveManufacturing}.

The design of support structures in additive manufacturing is thus an important consideration, with many factors affecting how the support structure will be built. A well optimized and design support structure will enable the part to be realized, while at the same time using the least amount of material possible to obtain a specified objective. Thermal and mechanical requirements are usually at the forefront of support structure design as they have a direct effect on the quality of the component, although there might be other considerations for the design, such as ease of removability, build time, and material efficiency. 

Thermal conductivity is an important factor since good thermal conduction in the support structures can improve the cooling process during fabrication, which helps prevent issues caused by excessive thermal deformation such as thermal residual stresses, thermal dilation, cracks or warping \cite{allaireOptimizingSupportsAdditive2018} \cite{zhouTopologyOptimizationThermal2019}. Additionally, large portions of surfaces that are almost horizontal and that are unsupported during manufacturing tend to be heavily distorted after manufacturing \cite{allaireOptimizingSupportsAdditive2018}. These regions, which are called overhang structures, typically require support structures that will reduce the deformations and prevent warpage of overhangs. At these regions, it is crucial that support  these  support structures also require to be as stiff as possible to prevent any distortion in the built part and to be able to withstand the weight of the part itself \cite{kuoSupportStructureDesign2018}. The support structures must then also be robust enough to withstand the weight in the part, especially in these overhang areas, to prevent distortion during the build process \cite{kumarTailoredSupportStructures2020}.

Topology optimization is a method that is suitable for the creation and optimization of shapes that satisfy specific constraints while realizing certain objectives \cite{bendsoeTopologyOptimization2002}, \todo{maybe add more references here} and is a well-known technique that has been obtaining more interest from the additive manufacturing community for both the design of the components themselves and for their support structures in additive manufacturing. \todo{maybe talk more about topology optimization here}. Topology optimization has been studied for designing parts that either have as little overhang surfaces as possible, thus requiring less supporting structures, or for designing parts that are self-supporting, thus requiring no supporting structures at all. Although this aid considerably in reducing the amount of material used, these approaches do not take into consideration other effects such as the thermal dissipation of the component, require changing the geometry of the component thus impacting its functionality \cite{yeTopologyOptimisationSelfsupporting2023}, or impose geometrical constraints that can restrict the component's performance \cite{langelaarTOPOLOGYOPTIMIZATIONADDITIVE2016}.

\todo{this paragraph talks about two uses for topology optimization: for the part itself, and for support strucutres.}
Topology optimization can be used for the design of the desired component itself, as to minimize and avoid the amount of overhang structures, or to make components that are self supporting.


Lee and Xie \cite{leeSimultaneouslyOptimizingSupports2021} developed an optimization algorithm that detects the best locations for supports in the boundaries of the design space, which can lead to an increase in stiffness and aid in minimizing deformation.

\section{Topology optimization}

Topology optimization is an optimization technique that seeks to find the optimal shape within a volume that satisfies certain governing equations while at the same time satisfying specific constraints. This technique is usually utilized for the design of structures with no preconceived shape. mathematically speaking, topology optimization seeks to find the optimal distribution of a design variable x within a design domain. The placement of x will also obey certain governing equations that are valid within the domain, and the existence of this distribution of x will also depend on a certain objective that is wished to be minimized, alongside other constraints that might be imposed in the syste. A classical example of topology optimization is the so called binary compliance problem, in which regions of solid and void material are distributed inside a volume to design a structural component that will be able to withstand certain loads applied to its boundaries, but that will have the least amount of deformation possible, while only using a certain fraction of the design volume, or keeping the total weight of the structure withing a certain limit. We can express this specific problem mathematically as:

\begin{align} 
  min: & \hspace{0.5cm} c(\bm{\rho}) = \bm{F}^T \bm{U}  \\
  s.t.: & \hspace{0.5cm} \bm{K}(\bm{\rho})\bm{U} = \bm{F}  \\
        & \hspace{0.5cm} V = \sum _{i \in \Omega} \rho_i v_i \leq V_c \\ 
        & \hspace{0.5cm} 0 \leq \rho_{min} \leq \rho_i \leq 1
\end{align}
\label{eq:topopt_eq}

where in equation \ref{eq:topopt_eq} the objective function that is to be minimized is given by $c(\rho) = f^T u$, where $\rho$ takes the value 0 or 1 depending on whether our small region in space is empty or contains material, $u$ is the displacement of piece of material and $f$ is the force applied to it. This quantity is referred to as compliance, and physicall it is the inverse of the stiffness of the structure. The deformation of this material in the design domain is controlled by Hooke's Law, which is shown in the above equation as $K(\rho) u = f$. $\bm{K}(\bm{\rho})$ is defined as a global stiffness matrix \todo{cite 19 Wilberg finite element method book}

\begin{equation}
  abc
  \label{eq:stiffness}
\end{equation}

where we sum over all the elements of the domain and $\bm{\rho}$ is a vector that contains the design variables \cite{lazarovFiltersTopologyOptimization2011}. 

Unfortunately, the formulation above suffers from a major problems. Stated as is, the problem above is well-known to be ill-posed \todo{cite Kohn and Strang, reference in Liu paper}, as it is possible to obtain a chattering design with an infinite number of holes of infinitesimal size, thus rendering this compliace problem to be unbounded \cite{liuEfficient3DTopology2014}. To remedy this situation, several approaches have been proposed in the literature. One approach to control the chattering design is ensure that the total perimeter of the resulting structure has un upper bound \cite{haberNewApproachVariabletopology1996} \cite{Jog 2002}, but this e method suffers from several complications in implementation, and small variations in the parameters of the algorithm can lead to wildly different designs of the final structure \cite{Jog 2002}.

A different alternative would be the utilization of a homogenization method \cite{Bendsoe 1995, Allaire 2001} \cite{suzukiHomogenizationMethodShape1991} in which the binary representation of the material within the design domain is relaxed and intermediate values of densities are allowed, instead of just allowing empty and filled-values. One of the difficulties with using the homogenization method is how to interpret the intermiediate densities of the material. In topology optimization problem involving the design of fluid flow media, the minimum value of the density could be interpreted as a fluid, the maximum value could be interpreted as a solid, while intermediate values could be interpreted as porous media \cite{pietropaoliThreedimensionalFluidTopology2019}. In structural problems, intermediate values could be interpreted as periodic composite materials with high-resolution microscopic features \cite{groenHomogenizationbasedTopologyOptimization2018} \cite{alexandersenTopologyOptimisationManufacturable2015}, materials that are composed of lattice structures \cite{allaireTopologyOptimizationModulated2019}, or even complex structures consisting of anisotropic fiber-reinforced composite materials \cite{kimTopologyOptimizationFunctionally2020}. Although the homogenization method is successful in solving the chattering design problem, validation of the resulting topologies using most FEM software are very computationally expensive, since the resulting microstructures requires very fine meshes with high number of elements and nodes \cite{kimComputationalHomogenizationAdditively2022}. Additionally, the use of microstructures can also lead to stress amplifications which need to be managed appropriately to avoid regions of high stress concentration that might compromise the stability or functionality of the manufactured part \cite{allaireTopologyOptimizationMinimum2004}.

Talk here about density-based approach and SIMP 

Talk here about the filters, density filters and Helmholtz filters

Talk here about hyperbolic tangent and what it does to the fitler


\section{Research Purpose}
 
This is something here

\section{Research originality and contribution}

\end{document}
