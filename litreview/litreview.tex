\documentclass[../main.tex]{subfiles}
\graphicspath{{\subfix{../images/}}}

\begin{document}
	
\chapter{Literature Review}

What I need to do in the literature review is an introduction, body and conclusion. The introduction should be an overview of the topics that I am going to talk about, and probably introduce the research motivation.

\subsection{introduction}

Additive manufacturing, also known as 3D printing, is a manufaacturing process in which parts are built by stacking layers of material on top of each other until the desired geometry is created. In particular, powdered-bed fusion is a type of additive manufacturing in which the base material is composed of a metallic powdered. A bed is then filled with this powder, and a high-powered laser or electron beam is utilized to selectively add heat to locations of this powdered metal, thus melting it. As the melted portions cool, it leaves a layer of solid material, and the whole component is then built layer by layer in this manner. If a laser is used for manufacturing, the process is then known as laser powder bed fusion (LPBF).

Additive manufacturing in general allows for the creation of parts with complex geometries that would otherwise be difficult, time-consuming ortoo costly to manufacture with other more traditional means of manufacturing such as machining or casting. This can be contrasted with the manufacture of prothesis using traditional manufacturing, wchih relies on the high-volume production of standarized parts and shapes. Due to the ease of creating parts with complex geometries and with shorter lead times, additive manufacturing has had a growing interest from the medical industry for the creation of custom made implants tailored to an individual patient's anatomy \cite{marshTrendsDevelopmentsHip2021} \\cite{narraAdditiveManufacturingTotal2019}. Additionally, this benefit of creating customized parts also leads to an improvement of patient comfort and better outcomes in orthopedic and surgical applications, as well as enhacing surgical precision and reducing complications of post-operative care \cite{mobarakRecentAdvancesAdditive2023} \cite{pathak3DPrintingBiomedicine2023}.

Nevertheless, even though additive manufacturing holds promise for the production of cuztomized medical implants, its cost can still remain high due to machine, material and process-level expenses, and thus its usage can still be limited only for revision procedures, i.e. second or third procedures in which the objective is to replace a failed or uncomfortable component \cite{narraAdditiveManufacturingTotal2019}. To solve the problem of high material costs, Laureijs et. al. propose that, in the case of powdered base additive manfucturing, one of the main driving costs is the price of the powder price. One of the viable options to drive down the price of additive manufacturing then would be to reduce the amount of scrap material involved in the process. Scrap material in additive manufacturing results from discarding the support structures that the part requires as well as any addition or removal of material to the part in post-processing steps to ensure that the part is maintained withing tolerances \cite{MetalAdditiveManufacturing}.

Then, need to talk about the different ways that support strutures could be designed in order to minimize the amount of material used, and maybe a bit about how good support strucutres can also impact the final part finish so that you dont need to use a lot of post processing to achieve tolerances.

\subsection{Body}

\subsubsection{Additive manufacturing}

This section needs to be more specific about how additive manufacturing works.1

\subsubsection{Support structures}

\subsubsection{Topology optimization}

Topology optimization is an optimization technique that seeks to find the optimal shape within a volume that satisfies certain governing equations, while at the same time satisfying specific constraints. This technique is usually utilized for the design of structures with no preconceived shape. When topology optimization is used for the purpose of designing structures, the optimal shape resulting from the analysis can be interpreted as a material distribution within the specified volume, which is called the design domain. A classical application of topology optimization is the binary compliance problem, in which regions of solid and void materials are distributed in order to minimize the work done by external forces (this is called compliance), also subject to a volume constraint \cite{liuEfficient3DTopology2014}. The problem of topology optimization can be expressed mathematically as follows \cite{lazarovFiltersTopologyOptimization2011}

\begin{equation}
  \label{eq:topopt_eq}
  min: c(\rho) = f^T u 
  s.t. : K(\rho)u = f 
  V = \Sigma i \in \Omega \rho_i v_i
\end{equation}

In the above formulation, the design domain 

This binary compliance problem suffers from a serious drawback; the problem is ill-posed and the solution will converge to a material distribution in the shape of a checkerboard pattern, with infinitesimal holes. To alleviate this, several restrictions must be made  and the equations used to generate the structure must be modified. 

Talk here about density-based approach and SIMP 

Talk here about the filters, density filters and Helmholtz filters

Techniques such as perimeter control methods [] density gradient control 
and mesh-independent density filters could be used to solve the problem of a chattering design [reference goes here, fitler in topology optimization. A filtering approach implements an algorithm that itereates over all the points in the design domain, and uses weighted values of its surroindings to determine the final density value of the point in question, effectively computing the weighted average of a point and its surroindings.

Talk here about hyperbolic tangent and what it does to the fitler

Here, the full equation of topology optimization using all of the above should be shown.

\section{Research Purpose}
 
This is something here

\section{Research originality and contribution}

\end{document}
