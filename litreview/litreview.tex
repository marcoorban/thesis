\documentclass[../main.tex]{subfiles}
\graphicspath{{\subfix{../images/}}}

\begin{document}
	
\chapter{Literature Review}

What I need to do in the literature review is an introduction, body and conclusion. The introduction should be an overview of the topics that I am going to talk about, and probably introduce the research motivation.

\section{Introduction}

Additive manufacturing, also known as 3D printing, is a manufaacturing process in which parts are built by stacking layers of material on top of each other until the desired geometry is created. In particular, powdered-bed fusion is a type of additive manufacturing in which the base material is composed of a metallic powdered. A bed is then filled with this powder, and a high-powered laser or electron beam is utilized to selectively add heat to locations of this powdered metal, thus melting it. As the melted portions cool, it leaves a layer of solid material, and the whole component is then built layer by layer in this manner. If a laser is used for manufacturing, the process is then known as laser powder bed fusion (LPBF).

Additive manufacturing in general allows for the creation of parts with complex geometries that would otherwise be difficult, time-consuming ortoo costly to manufacture with other more traditional means of manufacturing such as machining or casting. This can be contrasted with the manufacture of prothesis using traditional manufacturing, wchih relies on the high-volume production of standarized parts and shapes. Due to the ease of creating parts with complex geometries and with shorter lead times, additive manufacturing has had a growing interest from the medical industry for the creation of custom made implants tailored to an individual patient's anatomy \cite{marshTrendsDevelopmentsHip2021} \\cite{narraAdditiveManufacturingTotal2019}. Additionally, this benefit of creating customized parts also leads to an improvement of patient comfort and better outcomes in orthopedic and surgical applications, as well as enhacing surgical precision and reducing complications of post-operative care \cite{mobarakRecentAdvancesAdditive2023} \cite{pathak3DPrintingBiomedicine2023}.

Nevertheless, even though additive manufacturing holds promise for the production of cuztomized medical implants, its cost can still remain high due to machine, material and process-level expenses, and thus its usage can still be limited only for revision procedures, i.e. second or third procedures in which the objective is to replace a failed or uncomfortable component \cite{narraAdditiveManufacturingTotal2019}. To solve the problem of high material costs, Laureijs et. al. propose that, in the case of powdered base additive manfucturing, one of the main driving costs is the price of the powder price. One of the viable options to drive down the price of additive manufacturing then would be to reduce the amount of scrap material involved in the process. Scrap material in additive manufacturing results from discarding the support structures that the part requires as well as any addition or removal of material to the part in post-processing steps to ensure that the part is maintained withing tolerances \cite{MetalAdditiveManufacturing}.

The design of support structures in additive manufacturing is thus an important consideration, with many factors affecting how the support structure will be built. A well optimized and design support structure will enable the part to be realized, while at the same time using the least amount of material possible to obtain a specified objective. Thermal and mechanical requirements are usually at the forefront of support structure design as they have a direct effect on the quality of the component, although there might be other considerations for the design, such as ease of removability, build time, and material efficiency. 

Thermal conductivity is an important factor since good thermal conduction in the support structures can improve the cooling process during fabrication, which helps prevent issues caused by excessive thermal deformation such as thermal residual stresses, thermal dilation, cracks or warping \cite{allaireOptimizingSupportsAdditive2018} \cite{zhouTopologyOptimizationThermal2019}. Additionally, large portions of surfaces that are almost horizontal and that are unsupported during manufacturing tend to be heavily distorted after manufacturing \cite{allaireOptimizingSupportsAdditive2018}. These regions, which are called overhang structures, typically require support structures that will reduce the deformations and prevent warpage of overhangs. At these regions, it is crucial that support  these  support structures also require to be as stiff as possible to prevent any distortion in the built part and to be able to withstand the weight of the part itself \cite{kuoSupportStructureDesign2018}. The support structures must then also be robust enough to withstand the weight in the part, especially in these overhang areas, to prevent distortion during the build process \cite{kumarTailoredSupportStructures2020}.

Topology optimization is a method that is suitable for the creation and optimization of shapes that satisfy specific constraints while realizing certain objectives \cite{bendsoeTopologyOptimization2002}, \todo{maybe add more references here} and is a well-known technique that has been obtaining more interest from the additive manufacturing community for both the design of the components themselves and for their support structures in additive manufacturing. \todo{maybe talk more about topology optimization here}. Topology optimization has been studied for designing parts that either have as little overhang surfaces as possible, thus requiring less supporting structures, or for designing parts that are self-supporting, thus requiring no supporting structures at all. Although this aid considerably in reducing the amount of material used, these approaches do not take into consideration other effects such as the thermal dissipation of the component, require changing the geometry of the component thus impacting its functionality \cite{yeTopologyOptimisationSelfsupporting2023}, or impose geometrical constraints that can restrict the component's performance \cite{langelaarTOPOLOGYOPTIMIZATIONADDITIVE2016}.

\todo{this paragraph talks about two uses for topology optimization: for the part itself, and for support strucutres.}
Topology optimization can be used for the design of the desired component itself, as to minimize and avoid the amount of overhang structures, or to make components that are self supporting.


Lee and Xie \cite{leeSimultaneouslyOptimizingSupports2021} developed an optimization algorithm that detects the best locations for supports in the boundaries of the design space, which can lead to an increase in stiffness and aid in minimizing deformation.

\section{Additive manufacturing}

This section needs to be more specific about how additive manufacturing works.1

\section{Support structures}

\section{Topology optimization}

Topology optimization is an optimization technique that seeks to find the optimal shape within a volume that satisfies certain governing equations, while at the same time satisfying specific constraints. This technique is usually utilized for the design of structures with no preconceived shape. When topology optimization is used for the purpose of designing structures, the optimal shape resulting from the analysis can be interpreted as a material distribution within the specified volume, which is called the design domain. A classical application of topology optimization is the binary compliance problem, in which regions of solid and void materials are distributed in order to minimize the work done by external forces (this is called compliance), also subject to a volume constraint \cite{liuEfficient3DTopology2014}. The problem of topology optimization can be expressed mathematically as follows \cite{lazarovFiltersTopologyOptimization2011}

\begin{equation}
  \label{eq:topopt_eq}
  min: c(\rho) = f^T u 
  s.t. : K(\rho)u = f 
  V = \Sigma i \in \Omega \rho_i v_i
\end{equation}

In the above formulation, the design domain 

This binary compliance problem suffers from a serious drawback; the problem is ill-posed and the solution will converge to a material distribution in the shape of a checkerboard pattern, with infinitesimal holes. To alleviate this, several restrictions must be made  and the equations used to generate the structure must be modified. 

Talk here about density-based approach and SIMP 

Talk here about the filters, density filters and Helmholtz filters

Techniques such as perimeter control methods [] density gradient control 
and mesh-independent density filters could be used to solve the problem of a chattering design [reference goes here, fitler in topology optimization. A filtering approach implements an algorithm that itereates over all the points in the design domain, and uses weighted values of its surroindings to determine the final density value of the point in question, effectively computing the weighted average of a point and its surroindings.

Talk here about hyperbolic tangent and what it does to the fitler

Here, the full equation of topology optimization using all of the above should be shown.

\section{Heat transfer in supporting structures}

\section{Compliance in support structures}

\section{Research Purpose}
 
This is something here

\section{Research originality and contribution}

\end{document}
