\documentclass[../main.tex]{subfiles}
\graphicspath{{\subfix{../images/}}}

\begin{document}
	
\chapter{Literature Review}

\section{Additive manufacturing}

Additive manufacturing, also known as 3D printing, is a manufacturing process in which parts are built by stacking layers of material on top of each other, until the desired geometry is created. The actual process of additive manufacturing can be performed in various ways and with different materials. For many consumer-grade and hobby projects, additive manufacturing is usually done by squeezing material, usually a type of resin or plastic, from a nozzle, and building the part layer by layer. Apart from these, there is also the possibility of utilizing additive manufacturing for the production of metallic or ceramic pieces, in which the base material is in a powdered form. In powdered-bed fusion, a bed is filled with the metallic powder, and a laser is utilized to melt the powdered metal into the geometry that is required.

One of the greatest strengths of additive manufacturing is its capability of creating complex geometries that would otherwise be difficult, time-consuming, or costly using other methods of traditional manufacturing, such as machining or casting. Additive manufacturing allows for the production of intricately designed prostethics aligned to the anatomical needs of each patient, which leads to an improvement of fit and alignment, ultimately enhacing surgical outcome and patient satisfaction \cite{mobarakRecentAdvancesAdditive2023}. This is in direct contrast to traditional manufacturing, which relies on the high-volume production of standarized parts and shapes, and makes it really difficult to accomadate the needs of individual patients. Additionally, prosthetics created using addivitive manufacturing can be printed much more quickly, which reduces lead times and accelerates patient treatment, and enhance surgical precision which helps in reducing complications and shorteninig the hospital stays of patients \cite{pathak3DPrintingBiomedicine2023}.

Even though additive manufacturing can be more cost-efficient than traditional manufacturing for creating custom prosthetics, its usage can still be high for single patients, and thus its usage is limited only for revision arthroplasty procedures, which is a second surgery that seeks to replace a failed or uncomfortable component \cite{narraAdditiveManufacturingTotal2019}. The high cost of the knee replacement can be partly attributed to the high cost of the feedstock material. These costs can be minimized by using cheaper materials, but a better alternative to ensure that the patient utilizes the best material for his or her case would be to instead reduce the amount of material used. In particular, cost could be reduced by minimizing the amount of scrap material resulting from the manufacturing process. These scraps can include both the support structures for the component and any extra material added to the partto achieve the required tolerances \cite{MetalAdditiveManufacturing}.

\subsection{Powdered-bed fusion}

\subsection{Support structures}

\section{Topology Optimization}

Topology optimization is an optimization technique that seeks to find the optimal shape within a volume that satisfies certain governing equations, while at the same time satisfying specific constraints. This technique is usually utilized for the design of structures with no preconceived shape. When topology optimization is used for the purpose of designing structures, the optimal shape resulting from the analysis can be interpreted as a material distribution within the specified volume, which is called the design domain. A classical application of topology optimization is the binary compliance problem, in which regions of solid and void materials are distributed in order to minimize the work done by external forces (this is called compliance), also subject to a volume constraint \cite{liuEfficient3DTopology2014}. The problem of topology optimization can be expressed mathematically as follows \cite{lazarovFiltersTopologyOptimization2011}

\begin{equation}
  \label{eq:topopt_eq}
  min: c(\rho) = f^T u 
  s.t. : K(\rho)u = f 
  V = \Sigma i \in \Omega \rho_i v_i
\end{equation}

In the above formulation, the design domain 

This binary compliance problem suffers from a serious drawback; the problem is ill-posed and the solution will converge to a material distribution in the shape of a checkerboard pattern, with infinitesimal holes. To alleviate this, several restrictions must be made  and the equations used to generate the structure must be modified. 

Talk here about density-based approach and SIMP 

Talk here about the filters, density filters and Helmholtz filters

Techniques such as perimeter control methods [] density gradient control 
and mesh-independent density filters could be used to solve the problem of a chattering design [reference goes here, fitler in topology optimization. A filtering approach implements an algorithm that itereates over all the points in the design domain, and uses weighted values of its surroindings to determine the final density value of the point in question, effectively computing the weighted average of a point and its surroindings.

Talk here about hyperbolic tangent and what it does to the fitler

Here, the full equation of topology optimization using all of the above should be shown.

\section{Research Purpose}
 
This is something here

\section{Research originality and contribution}

\end{document}
