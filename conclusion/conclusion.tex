\documentclass[../main.tex]{subfiles}
\graphicspath{{\subfix{../images/}}}

\begin{document}

\chapter{Conclusions}

This is the conclusion

The conclusion is that topology optimization can help us reduce the volume of the supporting structure with minimal penalty to the overall deformation of the component. 


Additionally, the right topology can also help use minimize the maximum deformation of the component. Why this happens could be because there is less material that expands (the total delta of thermal expansion is proportional to the amount of material present, no?)


The study also shows that reducing the volume fraction too much might be a detriment to the thermal deformation. My hypothesis is that this occurs because heat has a harder time leaving the support structure since there aren’t as many good paths, which creates a high thermal gradient that makes the material expand more than usual.n

Additionally, from the simple geometry results, we can conclude that hyperbolic tangent angle is not directly correlated to the total deformation.


From the simple geometry results, we can also conclude that mechanical compliance has no effect on results. This is because the dominating effect on the physics of the system is thermal transfer only; the part is not heavy enough to create excessive deformation on the supporting structure.


Additionally, the judicious use of topology optimization with moderate volume fractions does not result in regions with stress greater than the maximum mechanical stress of the material, meaning that there is no fear of collapse.
\end{document}
