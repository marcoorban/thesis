\documentclass[../main.tex]{subfiles}

\begin{document}

\chapter*{Abstract}

 This study explores the application of topology optimization in designing support structures for components manufactured using additive manufacturing, with a focus on the impact of these structures on component deformation and stress. Utilizing the SIMP method coupled with hyperbolic tangent filtering and a multivalued objective function that considers both thermal and mechanical compliance, parametric studies were conducted to investigate the effects of varying hyperbolic tangent angles, volume fractions, objective function weights, and boundary conditions. The research centered on optimizing support structures for a femoral component used in total knee arthroplasty prostheses, as well as simpler geometric shapes. The findings indicate that mechanical compliance has a negligible effect on the objective function and can be disregarded. Furthermore, the results show that hyperbolic tangent angle, boundary conditions, and volume fraction have minimal influence on average nodal displacement and stress values. Notably, the simulation of the femoral component revealed a local minimum in maximum nodal deformation within the 40\% to 50\% volume fraction range, suggesting that topologies with volume fractions within this range may yield more favorable outcomes, while those with lower volume fractions may result in higher maximum deformations. These insights contribute to the understanding of topology optimization in additive manufacturing, particularly for complex components like those used in prosthetic devices.

Keywords: additive manufacturing, topology optimization, support structure design, parametric study, thermal displacement, residual stress

\end{document}
