\documentclass[../main.tex]{subfiles}
\graphicspath{{\subfix{../images/}}}

\begin{document}
	
\chapter{Introduction}
\section{Background}
 
 Ostheoarthritis is a very common type of arthritis that causes pain, swelling and stiffness in various body parts such as the hands, hips, back and knees. Over time, it affects bones, cartilage and other tissues, and is common in adults over the age of 45. Knee ostheoarthritis, also known as degenerate joint disease of the knee, is a type of ostheoarthritis that is predominantly seen in the elderly, and is a progressive disease that results in knee stiffness and swelling and pain after sitting or staying still for a long time. Although this disease can be treated, but not cured, with physical therapy and medications that slow down its progression, severe knee ostheoarthritis can only be resolved by means of surgery, in which the whole knee is replaced by a prothesis. Total knee arthoplasty (TKA), also called total knee replacement (TKR) is a very effective and consistently successful surgery that provides good outcomes for patients suffering from end-stage knee ostheoarthritis. TKR results in greatly improved pain relief and better quality of life for patients \cite{varacalloTotalKneeArthroplasty2025}.
 
Since ostheoarthritis is a disease that is most commonly seen in the elderly, as the percentage of elderly people in developed countries around the world increases, the prevention and treatment of diseases like ostheoarthritis become more important from a public health persepective. In particular, Taiwan has already  exceeded the threshold (14\%) of the definition of aged society established by the United Nations, with 3.983 million citizens over the age of 65, acounting for 17.18\% of the population \cite{ElderlyDisadvantagedSituation2024}. Additionally, it has been estimated that Taiwan's National Health Insurance already spens 5\% of its total expenditure on TKR every year, and the incidence of TKR has already tripled in the period between 1996 and 2010 \cite{linIncreaseTotalKnee2018}.
 
There has been a growing interest of using additivie manufacturing (AM) for the manufacture of the prosthetics used in TKA, since additive manufacturing is a technology well suited for the creation of custom, lightweight components with complex geometries, while also producing less material compared with other methods of manufacture \cite{narraAdditiveManufacturingTotal2019}. Even though additive manufacturing offers many advantages for the production of structures tailored to each individual patient, one of the major obstacles that stands in the way of a more widespread adoption are the high costs of AM. One way to reduce the cost of the total procedure is by reducing the amount of material used, which can also be accomplished by reducing the amount of scrap from the fabricated parts.  In additive manufacturing, much of the scrap comes from discarding the support structures that the components require for manufacturing, and thus making smaller support components or using smaller volumes for them would be a valid strategy to further reduce the cost of additive manufacturing components. The total deformation of the part after manufacture is also an important factor to consider, since additional costs can be incured from the addition or removal of material from the manufactured component by means of machining, in order to meet tolareances \cite{narraAdditiveManufacturingTotal2019}. 

\end{document}

