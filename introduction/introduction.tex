\documentclass[../main.tex]{subfiles}
\graphicspath{{\subfix{../images/}}}

\begin{document}
	
\chapter{Introduction}
\section{Nihao}
 
 \subsection{Background}
 
 Ostheoarthritis is a very common type of arthritis that causes pain, swelling and stiffness in various body parts such as the hands, hips, back and knees. Over time, it affects bones, cartilage and other tissues, and is common in adults over the age of 45. As percentage of elderly people in developed countries around the world increases, the prevention and treatment of diseases like ostheoarthritis become more important. In particular,  Taiwan has already  exceeded the threshold (14\%) of the definition of aged society established by the United Nations, with 3.983 million citizens over the age of 65, acounting for 17.18\% of the population \cite{ElderlyDisadvantagedSituation2024}. Total knee arthoplasty (TKA), also called total knee replacement (TKR) is a very effective and consistently successful surgery that provides good outcomes for patients suffering from end-stage knee ostheoarthritis, and serves to alleviate symptoms by totally replacing the damaged keen joint with prosthetic components. This results in greatly improved pain relief and better quality of life for patients \cite{varacalloTotalKneeArthroplasty2025}.
 
 Taiwan's National Health Insurance has spent 5\% of its total expenditure on TKR every year, and its incidence tripled between 1996 and 2010, making it an important public health issue.   according to \cite{linIncreaseTotalKnee2018}, and
 
There has been a growing interest of using additivie manufacturing (AM) for the manufacture of prosthetics used in TKA, since additive manufacturing is a technology well suited for the creation of custom, lightweight components with complex geometries, while also reducing material waste and keeping the weight of components light. \cite{narraAdditiveManufacturingTotal2019}. Even though additive manufacturing offers many advantages for the production of structures tailored to each individual patient, one of the major obstacles that stands in the way of a more widespread adoption are the high costs of AM. One way to reduce the cost of the total procedure is by reducing the amount of material used by minimizing the amount of scrap from the fabricated parts, which results from the support structures that the manufacturing process required as well as the addition or reduction of material to meet tolerances after machining \cite{narraAdditiveManufacturingTotal2019} \cite{MetalAdditiveManufacturing} . 

Additionally, the deviation between the native bone and the implant is crucial for postoperative comfort of the patient, as a deviation of just 1mm can cause an increased rate of wear between joints, which can result in postoperative infection, pain and instability \cite{chungpei-hsuYingYongMoNiFenXiYuShiYanSheJiFaTanTaoJingGeZhiChengJieGouYiZuiXiaoHuaXuanZeXingLeiSheRongRongZhiReBianXingStudyLattice2024}  
\cite{kebbachComputerbasedAnalysisDifferent2023}.
 
\end{document}

